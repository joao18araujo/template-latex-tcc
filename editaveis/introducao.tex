\chapter*[Introdução]{Introdução}
\addcontentsline{toc}{chapter}{Introdução}

\section{Contextualização}

Desde sua criação, o computador é utilizado para automatizar tarefas antes realizadas por meio de esforço humano. Seja calculando ou montando partes de um automóvel, tais máquinas são capazes de realizar o mesmo trabalho de modo mais eficiente e menos passível de erros. Contudo, uma barreira no uso de \textit{softwares} e sistemas para automação de atividades encontra-se nas artes. Devido a seu caráter criativo e emocional, por muito tempo, imaginou-se ser impossível para um computador produzir arte - seja ela visual ou sonora. Mas pode ser possível para uma máquina de calcular abstrair convenções de composição e compor peças musicais agradáveis aos ouvidos humanos (ref1).

A área de pesquisa responsável pelo estudo da composição musical automatizada é conhecida como composição algorítmica. Em um \textit{software} de composição algorítmica, este recebe parâmetros de entrada (como outras músicas, \textit{inputs} de usuário ou dados randômicos) e, por meio de um algoritmo que possui convenções de composição como restrição, uma melodia é devolvida como saída (ref2). Uma das áreas de composição musical passível de automatização é a composição de contrapontos musicais, melodias com qualidade harmônica coerente com a melodia principal.

\section{Objetivos}

\subsection{Objetivo Geral}

O objetivo geral deste trabalho é a implementação de um \textit{software} de composição de contrapontos palestrinianos. A aplicação será implementada em C++, terá como entrada uma melodia principal por meio de formatos digitais de representação de música como \textit{Lilypond} e \textit{MusicXML} e como saída uma melodia que serve como contraponto para a melodia dada, de acordo com as restrições definidas pelo usuário na interface do programa.

\subsection{Objetivos Específicos}

\begin{itemize}
\item Implementar um algoritmo de leitura e interpretação de formatos digitais de representação de música
\item Analisar os \textit{softwares} de composição algorítmica de contrapontos já existentes
\item Implementar um algoritmo de composição de contrapontos que respeite as restrições dos modos litúrgicos definidos
\end{itemize}

\section{Metodologia}

A implementação do \textit{software} utilizará a metodologia ágil \textit{Extreme Programming (XP)} como ciclo de vida de desenvolvimento, dado o escopo, o nível de definição dos requisitos e o tamanho da equipe. Sendo dividido em iterações que evoluem a aplicação de forma incremental. Além disso, será realizada uma revisão sistemática do tema a fim de se levantar o que já foi produzido e quais são as lacunas deixadas por eles, isso permitirá que o escopo do projeto seja definido de forma que traga algo relevante, baseado no que já foi feito.

\section{Estrutura do Documento}

Este documento divide-se em X capítulos. O primeiro capítulo aborda a fundamentação teórica necessária para o projeto, trazendo conceitos de Engenharia de \textit{Software} e da Teoria Musical. O segundo capítulo consiste em uma análise bibliográfica, levantando e analisando, por meio de uma revisão sistemática, o que já foi produzido academicamente em relação a algoritmos de composição de contraponto. O terceiro capitulo define a metodologia a ser utilizada, apresentando cronograma e a descrição específica das funcionalidades. O quinto capítulo aprezenta os resultados obtidos, apresentando os algoritmos e o \textit{software} desenvolvidos. O sexto e último capítulo traz as considerações finais, analisando o nível de sucesso do projeto e apresentando possibilidades de trabalhos futuros.
