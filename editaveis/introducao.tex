\chapter*[Introdução]{Introdução}
\addcontentsline{toc}{chapter}{Introdução}

  Desde sua criação, o computador é utilizado para automatizar tarefas antes realizadas por meio de esforço humano. Seja calculando ou montando partes de um automóvel, tais máquinas são capazes de realizar trabalhos de modo eficiente e com poucos erros. Contudo, uma barreira no uso de \textit{softwares} e sistemas para automação de atividades encontra-se nas artes. Devido a seu caráter criativo e emocional, por muito tempo, imaginou-se ser impossível para um computador produzir arte -- seja ela visual ou sonora. Mas é possível para uma máquina de calcular abstrair convenções de composição e compor peças musicais agradáveis aos ouvidos humanos (ref1)(refX).

  A área de pesquisa responsável pelo estudo da composição musical automatizada é conhecida como composição algorítmica. Um \textit{software} de composição algorítmica recebe parâmetros de entrada (como outras músicas, \textit{inputs} de usuário ou dados randômicos) e, por meio de um algoritmo que possui convenções de composição como restrição, devolve uma melodia como saída (ref2).

  Uma das áreas de composição musical passível de automatização é a composição de contrapontos musicais, melodias com qualidade harmônica coerente com a melodia principal(refContraponto). Sendo assim, este trabalho possui a seguinte questão de pesquisa: \textit{Dada uma melodia monofônica, é possível gerar contrapontos palestrinianos algoritmicamente?}

  \section*{Objetivos}

    \subsection*{Objetivo Geral}

      O objetivo geral deste trabalho é a implementação de um \textit{software} de composição de contrapontos palestrinianos. A aplicação será implementada em C++, terá como entrada uma melodia principal por meio de formatos digitais de representação de música, como \textit{Lilypond}\footnotemark \footnotetext{\url{http://lilypond.org/}} e \textit{MusicXML}\footnotemark \footnotetext{\url{https://www.musicxml.com/}}, e como saída uma melodia que serve como contraponto para a melodia dada, de acordo com as restrições definidas pelo usuário na interface do programa.
      % TODO: nota de rodapé com sites oficiais do Lilypond e MusicXML



      \subsection*{Objetivos Específicos}

        Os objetivos específicos deste trabalho são:

        \begin{itemize}
          \item implementar um algoritmo de leitura e interpretação de formatos digitais de representação de música;
          \item analisar os \textit{softwares} de composição algorítmica de contrapontos já existentes;
          \item implementar um algoritmo de composição de contrapontos que respeite as restrições dos modos litúrgicos definidos.
        \end{itemize}

  \section*{Estrutura do Documento}

    Este documento divide-se em 4 capítulos. O capítulo 1 aborda a fundamentação teórica necessária para o projeto, trazendo conceitos de Engenharia de \textit{Software} e da Teoria Musical. O capítulo 2 define a metodologia a ser utilizada, apresentando cronograma e a descrição específica das funcionalidades. O capítulo 3 apresenta os resultados obtidos: os algoritmos e o \textit{software} desenvolvidos. O capítulo 4 traz as considerações finais, analisando o nível de sucesso do projeto e apresentando possibilidades de trabalhos futuros.
