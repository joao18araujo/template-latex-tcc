\chapter[Fundamentação Teórica]{Fundamentação Teórica}

  \section[Teoria Musical]{Teoria Musical}

    \subsection[Elementos Musicais]{Elementos Musicais}

      Falar de melodia, harmonia, duração, ritmo e ligar com notas

      \subsubsection[Notas musicais]{Notas musicais}

        Figuras musicais (organizadas em) -> Pauta (altura definida por) -> Clave (indica qual) -> Nota (falar sobre distâncias) (pode ter) -> acidentes musicais -> representação dos acidentes / armadura de clave
                                              (possui) |-> tempo / ritmo

      \subsection[Intervalos]{Intervalos}

        Classificação quanti / quali -> quanti é isso -> quali é isso -> perfeitos / maior / menores -> aumentados e diminutos -> consonantes e dissonantes
      \subsection[Escalas]{Escalas}
        Definição -> maior e menor -> nome de cada nota (foco em tônica e dominante)
      \subsection[Modos Litúrgicos]{Modos Litúrgicos}
      \subsection[Contraponto]{Contraponto}

    \section[Paradigmas de Solução de Problema]{Paradigmas de Solução de Problema}
      Discutir vários paradigmas, citar alguns, ligar com CS e DP

      \subsection[Busca Completa]{Busca Completa}
        Definição -> exemplo -> aplicação no projeto
      \subsection[Programação Dinâmica]{Programação Dinâmica}
        Definição -> exemplo -> aplicação no projeto
