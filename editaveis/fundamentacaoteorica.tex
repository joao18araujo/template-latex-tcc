\chapter[Fundamentação Teórica]{Fundamentação Teórica}

  \section[Teoria Musical]{Teoria Musical}

    A música é a arte de combinar sons e silêncios. Segundo Med, a música é constituída, principalmente, por melodia, harmonia, ritmo e contraponto. A melodia caracteriza-se por sons em ordem sucessiva, enquanto a harmonia caracteriza-se por sons em ordem simultânea. O ritmo é a ordem e proporção do sons de uma música. Já o contraponto é definido por meio de duas melodias dispostas simultaneamente.

    Os sons são parte fundamental da música, cada som possui diversas características como altura (frequência do som que o torna mais grave ou mais agudo), duração, intensidade e timbre. Quando o som é regular (possui altura definida), ele pode ser expresso por meio de notação musical.

    \subsection[Notação Musical]{Notação Musical}

      As notas musicais são utilizadas para representar os sons de uma música. Há sete notas musicais - usualmente nomeadas Dó, Ré, Mi, Fá, Sol, Lá, Si ou, respectivamente, C, D, E, F, G, A, B - que podem variar de acordo com a oitava ou acidentes musicais empregados, para representação de tais modificações, elas são dispostas em uma pauta.

      FIGURA DAS NOTAS

      A pauta é utilizada para a representação de uma música, contendo linhas e espaços que definem a nota e em qual oitava está posicionada tendo como base a clave definida. Cada clave define uma nota específica e todas as outras notas da pauta são definidas a partir desta. Por exemplo, a clave de Sol é a mais comumente usada e define que a nota da linha 2 é o G4 (Sol da quarta oitava).

      FIGURA DA PAUTA COM CLAVE DE SOL

      % TODO: jeito certo de citar tabela
      Além da altura, cada som possui uma duração representada pela figura musical na pauta. No sistema atual, começa-se pela semibreve, que possui valor 1. A partir dela, a próxima subdivisão equivale à metade do tempo da anterior. Sendo assim, a mínima possui metade da duração da semibreve e possui valor 2, a semínima possui metade da duração da mínima e possui valor 4 e assim sucessivamente. As figuras, seus valores e durações estão representados na tabela abaixo.

      TABELA

      A duração em segundos de uma nota é definida pela sua figura e pelo andamento da música. O andamento define quantas notas de uma determinada figura de ritmo pode ser tocada em um espaço de tempo. Comumente, utiliza-se a quantidade de semínimas que podem ser tocadas durante um minuto para se definir o andamento de uma música.

      Em uma pauta, as figuras de ritmo são agrupadas em compassos. Cada compasso pode agrupar um determinado número de notas de acordo com sua estrutura, definida pela quantidade de figuras e a figura de ritmo usada como base. Por exemplo, um compasso 3/4 possui 3 como a quantidade de valores e 4 como a figura de ritmo, sendo assim, esse tipo de compasso é preenchido por três semínimas -- que possui valor 4. Desse modo, é possível também encaixar seis colcheias (que possui metade da duração de uma semínima) ou uma mínima e uma semínima em um compasso 3/4. Na música, a estrutura do compasso também define os tempos fortes e fracos da música. Em uma música 2/4, por exemplo, o primeiro tempo é forte (ou \textit{arsis}) e o segundo tempo é fraco (ou \textit{thesis}).

      FIGURA COM EXEMPLOS DE COMPASSO

      Outro elemento que pode ser representado na pauta são os acidentes musicais. Cada nota possui a distância relativa de um tom ou um semitom das notas adjacentes, como representado na figura abaixo. Essa distância calculada em tons e semitons também é utilizada para a definição de intervalos e pode ser modificada por meio de acidentes musicais. Existem dois tipos de acidentes musicais, o bemol e o sustenido. O bemol abaixa a nota original em um semitom, como exemplo, o Fá e o Sol possuem um tom entre eles, porém, o Fá e o Sol bemol possuem um semitom entre eles. Já o sustenido aumenta a nota original em um semitom. Outras variações desses acidentes são o dobrado bemol, que abaixa a nota em um tom, e o dobrado sustenido, que aumenta a nota em um tom.

      FIGURA COM BEMOL E SUSTENIDO EM PAUTA

      Esses acidentes podem ser pontuais ou recorrentes durante uma música. Quando são pontuais, eles são expressos ao lado da nota que modificam. Quando são recorrentes, são representados no início da clave ou do primeiro compasso que modificam. Enquanto a representação pontual modifica apenas um único som em uma oitava específica, a representação no início modifica qualquer som daquela mesma nota, e a esse conjunto de acidentes representados no início da pauta dá-se o nome de armadura de clave.

      FIGURA REPRESENTANDO ARMADURA DE CLAVE

    \subsection[Intervalos]{Intervalos}

      % TODO: trocar BEMOL por símbolo
      % TODO: citar tabela pelo número?
      A diferença de altura entre duas notas pode ser classificada por meio de intervalos. Os intervalos podem ser harmônicos (quando se compara duas notas simultâneas) e melódicos (quando se compara duas notas em sequência). Um intervalo melódico pode ser ascendente, se a primeira nota for mais grave que a segunda, e descendente, se a primeira nota for mais aguda que a segunda. Eles também são classificados quantitativamente e qualitativamente. A classificação qualitativa de um intervalo tem como base a quantidade de notas entre as duas notas, incluindo-as e ignorando acidentes musicais. Por exemplo, a classificação quantitativa do intervalo entre o G4 (Sol na quarta oitava) e o E5 (Mi na quinta oitava) é sexta (6ª), que é a mesma que o intervalo entre G\#4 (Sol sustenido na quarta oitava) e EBEMOL5 (Mi bemol na quinta oitava). Já a classificação qualitativa tem como base o número de tons e semitons entre as duas notas analisadas. A tabela X apresenta a relação entre as classificações qualitativas e quantitativas dos intervalos.

      TABELA COM INTERVALOS

      Cada intervalo pode ser classificado como consonante ou dissonante, uma classificação que define se o som provocado pelas duas notas soando seguidas ou em paralelo causam um efeito de repouso ou tensão, respectivamente. A classificação de um intervalo em relação ao seu efeito pode variar de acordo com a época ou estilo musical, nesse trabalho, será adotada a classificação correspondente às regras do contraponto modal do século XVI. (refContraponto)

      \subsubsection[Intervalos Justos]{Intervalos Justos}

        % TODO: colocar abreviações nos primeiros exemplos de cada (1ªJ ou P1, 2ªM ou M2, etc)
        Os intervalos de primeira, quarta, quinta e oitava podem ser classificados como justos dependendo da distância em semitons.

        Uma nota é a primeira justa de outra se elas tiverem a mesma altura, ou seja, apenas uma nota entre elas e nenhum semitom de diferença, esse intervalo também é chamado de uníssono.

        FIGURA PRIMEIRA JUSTA

        A quarta justa ocorre quando há quatro notas entre elas e a diferença é de dois tons e um semitom.

        FIGURA QUARTA JUSTA

        A quinta justa ocorre quando há cinco notas entre elas e a diferença é de três tons e um semitom.

        FIGURA QUINTA JUSTA

        A oitava justa ocorre quando há oito notas entre elas e a diferença é de cinco tons e dois semitons.

        FIGURA OITAVA JUSTA

        Dentre os intervalos justos, todos, exceto a quarta justa, são considerados consonâncias perfeitas. Vale ressaltar que a quarta justa é considerada um intervalo consonante na música contemporânea. (refBohumil)

      \subsubsection[Intervalos Maiores e Menores]{Intervalos Maiores e Menores}

        Os intervalos de segunda, terça, sexta e sétima podem ser classificados como maiores ou menores dependendo da distância em semitons.

        A segunda ocorre quando há duas notas entre elas. A segunda menor ocorre quando a diferença é de um semitom, já a segunda maior possui uma diferença de um tom.

        FIGURA SEGUNDA MENOR E MAIOR


        A terça ocorre quando há três notas entre elas. A terça menor ocorre quando a diferença é de um tom e um semitom, já a terça maior possui uma diferença de dois tons.

        FIGURA TERÇA MENOR E MAIOR

        A sexta ocorre quando há seis notas entre elas. A sexta menor ocorre quando a diferença é de três tons e dois semitons, já a sexta maior possui uma diferença de quatro tons e um semitom.

        FIGURA SEXTA MENOR E MAIOR

        A sétima ocorre quando há sete notas entre elas. A sétima menor ocorre quando a diferença é de quatro tons e dois semitons, já a sétima maior possui uma diferença de cinco tons e um semitom.

        FIGURA SÉTIMA MENOR E MAIOR

        Os intervalos de terça maior e menor e sexta maior e menor são tidos como consonâncias imperfeitas, enquanto os intervalos de segunda maior e menor e sétima maior e menor são tidos como dissonâncias.

      \subsubsection[Intervalos Aumentados e Diminutos]{Intervalos Aumentados e Diminutos}

        Se a distância em semitons das duas notas do intervalo não identificá-lo como justo, maior ou menor, ele será classificado como aumentado ou diminuto.

        Intervalos aumentados possuem mais semitons que os intervalos justos ou maiores. Se o intervalo possuir um semitom a mais, ele é chamado de aumentado, se forem dois semitons a mais, é chamado de superaumentado, se forem três semitons a mais, é chamado de três vezes aumentado e assim sucessivamente, até os maiores intervalos possíveis que são os cinco vezes aumentados.

        FIGURA INTERVALOS AUMENTADOS

        Intervalos diminutos possuem menos semitons que os intervalos justos ou menores. Se o intervalo possuir um semitom a menos, ele é chamado de diminutos, se forem dois semitons a menos, é chamado de superdiminuto, se forem três semitons a menos, é chamado de três vezes diminuto e assim sucessivamente, até os menores intervalos possíveis que são os cinco vezes diminutos.

        FIGURA INTERVALOS DIMINUTOS

        Os intervalos aumentados e diminutos são classificados como consonantes e dissonantes de acordo com os intervalos justos, maiores e menores que possuem a mesma distância em semitons, por exemplo, o intervalo de quarta superaumentada é considerado consonante pois possui a mesma distância em semitons que a quinta justa, embora sua classificação quantitativa seja a mesma que a quarta justa, classificada como dissonância. Por causa disso, os intervalos aumentados e diminutos são considerados dissonâncias condicionais.

      \subsubsection[Intervalos Compostos]{Intervalos Compostos}

        Se um intervalo ultrapassar a oitava justa, quantitativa ou qualitativamente, ele é considerado composto. Intervalos compostos tem sua classificação definida de acordo com seus correspondentes simples. Para classificar um intervalo composto, divide-se o valor quantitativo por 7 e o resto é utilizado para definir seu equivalente simples, por exemplo, uma 15ª corresponde a uma 1ª. Para a análise quantitativa, divide-se o valor de semitons por 12 e o resto é utilizado para definir se ele é justo, maior, menor, diminuto ou aumentado, por exemplo uma 10ª com uma diferença de 16 semitons é equivalente a uma 3ª com 4 semitons, sendo uma 3ª maior, logo, o intervalo analisado é uma 10ª maior. O intervalo de oitava aumentada é considerado composto e é equivalente ao intervalo de primeira aumentada.

        FIGURA DÉCIMA AUMENTADA

    \subsection[Escalas]{Escalas}

      Segundo Med, escala é o conjunto de notas disponíveis em um sistema musical. As notas presentes em uma escala possuem intervalos específicos entre si. As escalas podem ser classificadas quanto ao número de notas, sendo as mais conhecidas a pentatônica (cinco notas), a hexacordal (seis notas), a heptatônica (sete notas) e a cromática (doze notas). A escala cromática, por exemplo, possui todas as dozes notas (com e sem acidentes) de uma oitava, tendo intervalos de segunda menor ou primeira aumentada entre elas.

      FIGURA ESCALA CROMÀTICA

      % TODO: como citar figura?
      Outra classificação é quanto à utilização, as mais conhecidas são as escalas classificadas como diatônicas, que possuem oito notas, sete distintas e a última sendo a primeira uma oitava acima, possuindo intervalos de segunda maior ou menor entre elas. Em uma escala diatônica, pode-se definir quais notas fazem parte dela observando a armadura de clave presente na partitura. Por exemplo, na escala representada na figura X, há uma armadura de sustenindo em F (Fá) em D (Dó), logo, a escala é possui as seguintes notas: D, E, F\#, G, A, B e C\#. Músicas compostas nessa escala não podem possuir notas como ABEMOL, B\# ou C, a não ser em caso de acidentes pontuais, definidos pelo compositor. Cada nota de uma escala diatônica possui um grau (que varia de um a sete) e um nome específicos. A principal nota é a de grau I chamada tônica, ela dá nome à escala e a inicia, tendo todas as outras notas definidas com base nela.

      FIGURA ESCALA RÉ MAIOR

      \subsubsection[Escalas Maiores]{Escalas Maiores}

        As escalas diatõnicas podem ser classificadas em dois modos: maior e menor. As escalas maiores iniciam na tônica e possuem os seguintes intervalos, em tons, entre notas consecutivas: tom - tom - semitom - tom - tom - tom - semitom. Classificando os intervalos, são: 2ªM - 2ªM - 2ªm - 2ªM - 2ªM - 2ªM - 2ªm. Na figura X, estão representadas as escalas de Dó maior (que não possui nenhum acidente musical em sua armadura) e a de Fá maior (que possui um bemol em Si em sua armadura).

        FIGURA ESCALAS DÓ MAIOR E RÉ MAIOR

      \subsubsection[Escalas Menores]{Escalas Menores}

        Assim como as escalas maiores, as escalas menores iniciam na tônica e possuem os seguintes intervalos, em tons, entre notas consecutivas: tom - semitom - tom - tom - semitom - um tom e meio - semitom. Classificando os intervalos, são: 2ªM - 2ªM - 2ªm - 2ªM - 2ªM - 2ªM - 2ªm. Na figura X, estão representadas as escalas de Lá menor (que não possui nenhum acidente musical em sua armadura) e a de Mi menor (que possui um sustenido em Fá em sua armadura).

        FIGURA ESCALAS LÁ MAIOR E MI MAIOR

        As escalas são utilizadas na composição de contrapontos definindo quais notas podem ser utilizadas para a geração do contraponto, de acordo com a escala da melodia principal.

    % \subsection[Modos Litúrgicos]{Modos Litúrgicos}
    % TODO: falar ou não de modos litúrgicos?

    \subsection[Contraponto]{Contraponto}

      Contraponto é a arte de tornar independentes linhas melódicas de expressão autônoma (refKoellreutter), ou seja, combinar duas linhas melódicas simultâneas de forma que sejam melódicamente independentes, mas harmonicamente interdependentes. O objeto de estudo desse trabalho é o contraponto modal do século XVI, também chamado de contraponto palestriniano. Esse tipo de contraponto possui diversas espécies, cada uma definida pelo número de notas que um contraponto deve ter para cada nota da melodia principal (ou \textit{cantus firmus}). Nesse trabalho, serão tratados os contrapontos de primeira e segunda espécie.

      \subsubsection[Contraponto de Primeira Espécie]{Contraponto de Primeira Espécie}

        O contraponto de primeira espécie possui as seguintes regras:

        \begin{itemize}
          \item Deve haver uma nota no contraponto para cada nota do \textit{cantus firmus}
          \item Os intervalos harmônicos entre o contraponto e o \textit{cantus firmus} devem ser consonates
          \item Deve-se começar e terminar em consonâncias perfeitas
          \item Se o contraponto for inferior (suas notas são mais graves que a do \textit{cantus firmus}), somente a oitava justa e o uníssono podem ser usados no início e no final
          \item O uníssono só pode ocorrer no início ou no final
          \item Não é permitido oitavas e quintas paralelas (dois ou mais intervalos de oitavas ou de quintas seguidos)
          \item Os intervalos não podem ser maiores que uma décima maior (10ªM)
          \item Só são permitidos, no máximo, quatro intervalos seguidos de terças ou sextas (maiores ou menores) seguidos
          \item Se o movimento for paralelo, os intervalos melódicos devem ser menores que uma quarta ou um salto de oitava
          \item Deve-se priorizar movimento contrário (utilizar intervalo melódico ascendente se o \textit{cantus firmus} estiver em intervalo melódico descendente e vice-versa)
        \end{itemize}

        FIGURA EXEMPLO SIMPLES CONTRAPONTO PRIMEIRA ORDEM

      \subsubsection[Contraponto de Segunda Espécie]{Contraponto de Segunda Espécie}

        O contraponto de segunda espécie possui, além das regras do contraponto de primeira espécie, as seguintes regras:

        \begin{itemize}
          \item Dissonâncias podem ser utilizadas em \textit{arsis} (porções não-acentuadas do compasso)
          \item As dissonâncias podem ser utilizadas apenas como nota de passagem, entre duas consonâncias e formando um grau conjunto, isto é, o intervalo melódico entre a nota anterior e posterior à nota que provoca dissonância deve ser de segunda, maior ou menor.
          \item O uníssono só pode ser utilizado no início, final ou em \textit{arsis}
          \item Se o uníssono for alcançado por salto (intervalo maior que uma segunda), deve ser deixado em grau conjunto na direção oposta à do salto
          \item Evitar quintas e oitavas em \textit{thesis} sucessivas, por se aproximarem de quintas e oitavas paralelas
        \end{itemize}

        FIGURA EXEMPLO SIMPLES CONTRAPONTO SEGUNDA ORDEM

  \section[Paradigmas de Solução de Problema]{Paradigmas de Solução de Problema}

    Discutir vários paradigmas, citar alguns, ligar com CS e DP

    \subsection[Busca Completa]{Busca Completa}
      Definição -> exemplo -> aplicação no projeto
    \subsection[Programação Dinâmica]{Programação Dinâmica}
      Definição -> exemplo -> aplicação no projeto
