\chapter[Fundamentação Teórica]{Fundamentação Teórica}

  \section[Teoria Musical]{Teoria Musical}

    A música é a arte de combinar sons e silêncios. Segundo Bohumil, a música é constituída, principalmente, por melodia, harmonia, ritmo e contraponto. A melodia caracteriza-se por sons em ordem sucessiva, enquanto a harmonia caracteriza-se por sons em ordem simultânea. O ritmo é a ordem e proporção do sons de uma música. Já o contraponto é definido por meio de duas melodias dispostas simultaneamente.

    Os sons são parte fundamental da música, cada som possui diversas características como altura (frequência do som que o torna mais grave ou mais agudo), duração, intensidade e timbre. Quando o som é regular (possui altura definida), ele pode ser expresso por meio de notação musical.

    \subsection[Notação Musical]{Notação musical}

      As notas musicais são utilizadas para representar os sons de uma música. Há sete notas musicais - usualmente nomeadas dó, ré, mi, fá, sol, lá, si ou, respectivamente, C, D, E, F, G, A, B - que podem variar de acordo com a oitava ou acidentes musicais empregados, para representação de tais modificações, elas são dispostas em uma pauta.

      FIGURA DAS NOTAS

      A pauta é utilizada para a representação de uma música, contendo linhas e espaços que definem a nota e em qual oitava está posicionada tendo como base a clave definida. Cada clave define uma nota específica e todas as outras notas da pauta são definidas a partir desta. Por exemplo, a clave de sol é a mais comumente usada e define que a nota da linha 2 é o G4 (sol da quarta oitava).

      FIGURA DA PAUTA COM CLAVE DE SOL

      % TODO: jeito certo de citar tabela
      Além da altura, cada som possui uma duração representada pela figura musical na pauta. No sistema atual, começa-se pela semibreve, que possui valor 1. A partir dela, a próxima subdivisão equivale à metade do tempo da anterior. Sendo assim, a mínima possui metade da duração da semibreve e possui valor 2, a semínima possui metade da duração da mínima e possui valor 4 e assim sucessivamente. As figuras, seus valores e durações estão representados na tabela abaixo.

      TABELA

      A duração em segundos de uma nota é definida pela sua figura e pelo andamento da música. O andamento define quantas notas de uma determinada figura de ritmo pode ser tocada em um espaço de tempo. Comumente, utiliza-se a quantidade de semínimas que podem ser tocadas durante um minuto para se definir o andamento de uma música.

      Em uma pauta, as figuras de ritmo são agrupadas em compassos. Cada compasso pode agrupar um determinado número de notas de acordo com sua estrutura, definida pela quantidade de figuras e a figura de ritmo usada como base. Por exemplo, um compasso 3/4 possui 3 como a quantidade de valores e 4 como a figura de ritmo, sendo assim, esse tipo de compasso é preenchido por três semínimas -- que possui valor 4. Desse modo, é possível também encaixar seis colcheias (que possui metade da duração de uma semínima) ou uma mínima e uma semínima em um compasso 3/4. Na música, a estrutura do compasso também define os tempos fortes e fracos da música. Em uma música 2/4, por exemplo, o primeiro tempo é forte (ou \textit{arsis}) e o segundo tempo é fraco (ou \textit{thesis}).

      FIGURA COM EXEMPLOS DE COMPASSO

      Outro elemento que pode ser representado na pauta são os acidentes musicais. Cada nota possui a distância relativa de um tom ou um semitom das notas adjacentes, como representado na figura abaixo. Essa distância calculada em tons e semitons também é utilizada para a definição de intervalos e pode ser modificada por meio de acidentes musicais. Existem dois tipos de acidentes musicais, o bemol e o sustenido. O bemol abaixa a nota original em um semitom, como exemplo, o fá e o sol possuem um tom entre eles, porém, o fá e o sol bemol possuem um semitom entre eles. Já o sustenido aumenta a nota original em um semitom. Outras variações desses acidentes são o dobrado bemol, que abaixa a nota em um tom, e o dobrado sustenido, que aumenta a nota em um tom.

      FIGURA COM BEMOL E SUSTENIDO EM PAUTA

      Esses acidentes podem ser pontuais ou recorrentes durante uma música. Quando são pontuais, eles são expressos ao lado da nota que modificam. Quando são recorrentes, são representados no início da clave ou do primeiro compasso que modificam. Enquanto a representação pontual modifica apenas um único som em uma oitava específica, a representação no início modifica qualquer som daquela mesma nota, e a esse conjunto de acidentes representados no início da pauta dá-se o nome de armadura de clave.

      FIGURA REPRESENTANDO ARMADURA DE CLAVE

      \subsection[Intervalos]{Intervalos}

        Classificação quanti / quali -> quanti é isso -> quali é isso -> perfeitos / maior / menores -> aumentados e diminutos -> consonantes e dissonantes
      \subsection[Escalas]{Escalas}
        Definição -> maior e menor -> nome de cada nota (foco em tônica e dominante)
      % \subsection[Modos Litúrgicos]{Modos Litúrgicos}
      % TODO: falar ou não de modos litúrgicos?
      \subsection[Contraponto]{Contraponto}

    \section[Paradigmas de Solução de Problema]{Paradigmas de Solução de Problema}
      Discutir vários paradigmas, citar alguns, ligar com CS e DP

      \subsection[Busca Completa]{Busca Completa}
        Definição -> exemplo -> aplicação no projeto
      \subsection[Programação Dinâmica]{Programação Dinâmica}
        Definição -> exemplo -> aplicação no projeto
