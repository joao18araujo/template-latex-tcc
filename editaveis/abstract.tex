\begin{resumo}[Abstract]
 \begin{otherlanguage*}{english}
   Algorithmic composition is the creation of music through computational algorithms, focusing on automation and unpredictability, usage of this type of algorithm has several applications. The goal of this project is to develop a software capable of composing counterpoint in the Palestrina style algorithmically to a given melody. This development was based on musical theory and Palestrina style counterpoint rules study, followed by requirements definition and development of a prototype with musical notation, intervals and scales modules that should be capable of generating first species counterpoints. The final stage of development consisted of testing previously implemented modules, implementation of modules to generate second to fourth species counterpoint and development of a MIDI generation module that returns the MIDI of the original song with the counterpoint. The application has a solution that uses an implicit graph along with DFS and dynamic programming. The developed software is capable of reading a full Lilypond file, store its melody, store and use intervals and scales, generating counterpoints of first to second species and sending to the user a compacted file containing the Lilypond file, the MIDI file and the PDF with the sheet.

   \vspace{\onelineskip}

   \noindent
   \textbf{Key-words}: algorithmic composition, complete search, graphs, dfs, dp, dynamic programming, palestrina style counterpoints.
 \end{otherlanguage*}
\end{resumo}
