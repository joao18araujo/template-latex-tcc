\begin{resumo}[Abstract]
 \begin{otherlanguage*}{english}
   Algorithmic composition is the creation of music through computational algorithms, focusing on automation and unpredictability, usage of this type of algorithm has several applications. The goal of this project is to develop a software capable of composing counterpoint in the Palestrina style algorithmically to a given melody. This development was based on musical theory and Palestrina style counterpoint rules study, followed by requirements definition and development of a prototype with musical notation, intervals and scales modules that should be capable of generating first species counterpoints and serve as a guide to development of second to fifth counterpoint modules. Initially, a complete search algorithm was used and later changed to a solution that uses an implicit graph along with DFS and dynamic programming. The developed prototype is capable of reading notes on Lilypond format and store them, store and use intervals and scales, generating first species counterpoint.

   \vspace{\onelineskip}

   \noindent
   \textbf{Key-words}: algorithmic composition, complete search, graphs, dfs, dp, dynamic programming, palestrina style counterpoints.
 \end{otherlanguage*}
\end{resumo}
