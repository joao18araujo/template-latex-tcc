\chapter[Resultados Obtidos]{Resultados Obtidos}

  Na primeira parte desse trabalho, foi implementado um protótipo com as condições mínimas para a execução do algoritmo de composição de contrapontos de primeira espécie. Para isso, foram implementados os módulos de notação musical, intervalos, escalas e contraponto.

  \section[Módulo de Notação Musical]{Módulo de Notação Musical}

    O módulo de notação musical possui duas classes: \textit{Note} e \textit{Note Reader}.

    \subsection[\textit{Note}]{\textit{Note}}

      A classe \textit{Note} é responsável pelo armazenamento das informações de uma nota. Os atributos dos objetos dessa classe são:

      \begin{enumerate}
        \item \textit{Duration}: Número inteiro que representa a quantidade de tempo que a nota é tocada dentro de um compasso.
        \item \textit{Absolute Time}: Número inteiro que indica a posição absoluta da nota na melodia, importante para contrapontos de segunda espécie em diante que utilizam da posição da nota no compasso.
        \item \textit{Note}: \textit{String} que armazena qual nota musical, entre A e G, ou R quando for uma pausa.
        \item \textit{Accidental}: \textit{String} que armazena os acidentes musicais que ocorrem à nota, se houver.
        \item \textit{Octave}: Número inteiro que indica qual de qual oitava a nota é.
        \item \textit{MIDI Number}: Número inteiro que representa a nota no formato padronizado MIDI, por exemplo, o Dó na quarta oitava é representado pelo número 72.
        \item \textit{Note Number}: Número inteiro que representa a nota desconsiderando acidentes, utilizado para a classificação quantitativa de intervalos.
        \item \textit{Valid}: Atributo booleano que indica se a nota é válida não, útil para retornar notas inválidas sem utilizar ponteiros, quando necessário.
      \end{enumerate}

      Além de construtores, a classe possui os seguintes métodos:

      \begin{enumerate}
        \item \textit{Full Note}: Retorna uma \textit{string} representando a nota.
        \item \textit{Set Full Note}: Constrói os outros atributos da uma nota a partir de uma \textit{string} padronizada, utilizada em construtores e outros métodos.
        \item \textit{Enarmonies}: Retorna um vetor com as notas enarmônicas à nota do objeto.
      \end{enumerate}

      Por ser um protótipo, todos os atributos e métodos são públicos, isso será mudado na versão final, a Figura \ref{noteclass} descreve os atributos, métodos e construtores da classe.

      FIGURA CLASSE NOTA

    \subsection[\textit{Note Reader}]{\textit{Note Reader}}
  \section[Módulo de Intervalos]{Módulo de Intervalos}
    \subsection[\textit{Interval}]{\textit{Interval}}
  \section[Módulo de Escalas]{Módulo de Escalas}
    \subsection[\textit{Scale}]{\textit{Scale}}
  \section[Módulo de Contraponto]{Módulo de Contraponto}
    \subsection[\textit{Counterpoint}]{\textit{Counterpoint}}
    \subsection[\textit{First Order Counterpoint}]{\textit{First Order Counterpoint}}

  \section[Experimentos]{Experimentos}
  \subsection[\textit{Solução Ad-hoc}]{Solução Ad-hoc}
  \subsection[\textit{Solução com Busca Completa}]{Solução com Busca Completa}
    \subsection[\textit{Solução com DP}]{Solução com DP}
