\chapter[Resultados Obtidos]{Resultados Obtidos}

  Na primeira parte desse trabalho, foi implementado um protótipo com as condições mínimas para a execução do algoritmo de composição de contrapontos de primeira espécie. Para isso, foram implementados os módulos de notação musical, intervalos, escalas e contraponto.

  \section[Módulo de Notação Musical]{Módulo de Notação Musical}

    O módulo de notação musical possui quatro classes: \textit{Note}, \textit{Note Reader}, \textit{Compass Time} e \textit{Song}.

    \subsection[\textit{Note}]{\textit{Note}}

      A classe \textit{Note} é responsável pelo armazenamento das informações de uma nota. Os atributos de instância dessa classe são:

      \begin{enumerate}
        \item \textit{Duration}: número inteiro que representa a quantidade de tempo que a nota é tocada dentro de um compasso.
        \item \textit{Absolute Time}: número inteiro que indica a posição absoluta da nota na melodia, importante para contrapontos de segunda espécie em diante que utilizam da posição da nota no compasso.
        \item \textit{Note}: \textit{string} que armazena qual nota musical, entre A e G, ou R quando for uma pausa.
        \item \textit{Accidental}: \textit{string} que armazena os acidentes musicais que ocorrem à nota, se houver.
        \item \textit{Octave}: número inteiro que indica qual de qual oitava a nota é.
        \item \textit{MIDI Number}: número inteiro que representa a nota no formato padronizado MIDI, por exemplo, o Dó na quarta oitava é representado pelo número 72.
        \item \textit{Note Number}: número inteiro que representa a nota desconsiderando acidentes, utilizado para a classificação quantitativa de intervalos.
        \item \textit{Valid}: atributo booleano que indica se a nota é válida não, útil para retornar notas inválidas sem utilizar ponteiros, quando necessário.
      \end{enumerate}

      Além de construtores, a classe possui os seguintes métodos:

      \begin{enumerate}
        \item \textit{Full Note}: retorna uma \textit{string} representando a nota.
        \item \textit{Set Full Note}: constrói os outros atributos da uma nota a partir de uma \textit{string} padronizada, utilizada em construtores e outros métodos.
        \item \textit{Enarmonies}: retorna um vetor com as notas enarmônicas à nota do objeto.
      \end{enumerate}

      Por ser um protótipo, todos os atributos e métodos são públicos, isso será mudado na versão final, a Figura \ref{noteclass} descreve os atributos, métodos e construtores da classe.

      FIGURA CLASSE NOTA

    \subsection[\textit{Note Reader}]{\textit{Note Reader}}


      A classe \textit{Note Reader} é responsável pelo \textit{parser} de \textit{string} representado notas em formato Lilypond para instâncias da classe \textit{Note} e vice-versa. Sem a necessidade de atributos e com apenas métodos estáticos, a classe age como uma biblioteca, sendo utilizada sem instanciação. Ela possui os seguintes métodos:


      \begin{enumerate}
        \item \textit{String To Note}: retorna uma instância de \textit{Note} ao receber uma \textit{string} e uma nota. A nota recebida como parâmetro é a nota anterior, pois o formato Lilypond omite a duração da nota se for a mesma da anterior.
        \item \textit{Note To String}: retorna uma \textit{string} que representa uma nota em formato Lilypond ao receber um objeto \textit{Note}.
        \item \textit{MSB}: retorna o bit mais significativo, utilizado no cálculo da figura musical baseado na duração da nota. Quanto mais for o valor da figura musical, menor a duração, mas pontos de aumento podem modificar a duração da nota e o bit mais significativo é usado para inferir o valor da figura musical, pois as figuras sem pontos de aumento sempre geram durações representadas por potências de 2.
        \item \textit{Number Of On Bits}: retorna o número de bits ligados de um inteiro, útil pois deriva o número de pontos de aumento de uma nota (para gerar a \textit{string}) a partir da representação binária da duração.
      \end{enumerate}

      A Figura \ref{notereaderclass} representa os métodos da classe.

      FIGURA CLASSES NOTE READER

    \subsection[\textit{Compass Time}]{\textit{Compass Time}}

      A classe \textit{Compass Time} armazena o tempo de compasso de uma música. Possui os seguintes atributos:

      \begin{enumerate}
        \item \textit{Times}: Número inteiro, armazena a quantidade de batidas que o compasso possui, por exemplo, em um compasso 3/4, ele armazena o valor 3.
        \item \textit{Base Note}: Número inteiro, armazena qual é a figura musical base do compasso, por exemplo, em um compasso 3/4, ele armazena o valor 4.
      \end{enumerate}

      Além dos atributos, essa classe possui dois métodos:

      \begin{enumerate}
        \item \textit{Base Note Duration}: Retorna a duração da nota base.
        \item \textit{Compass Duration}: Retorna a duração total do compasso, multiplicando a duração da nota base pela quantidade de batidas.
      \end{enumerate}


      A Figura \ref{compasstimeclass} representa os atributos e métodos da classe.

      FIGURA CLASSES COMPASS TIME

    \subsection[\textit{Song}]{\textit{Song}}

      A classe \textit{Song} armazena as informações básicas de uma música. Possui os seguintes atributos:

      \begin{enumerate}
        \item \textit{Notes}: Vetor que armazena as notas de uma música em ordem de execução.
        \item \textit{Scale}: Armazena a escala de uma música.
        \item \textit{Compass Time}: Armazena o tempo de compasso de uma música, assumindo que não há troca de tempo no meio da canção.
      \end{enumerate}

      Ela possui apenas um método chamado \textit{Size} que retorna o número de notas da música.


  \section[Módulo de Intervalos]{Módulo de Intervalos}

    Responsável pela geração e armazenamento de intervalos, este módulo possui a classe \textit{Interval}. A Figura \ref{songclass} representa os atributos e métodos da classe.

    FIGURA CLASSES SONG

    \subsection[\textit{Interval}]{\textit{Interval}}

      A classe \textit{Interval} armazena intervalos e possui método para gerar uma nota a partir de outra e um intervalos. Ela possui os seguintes atributos:

      \begin{enumerate}
        \item \textit{Quantitative}: número inteiro, representa a classificação quantitativa do intervalo.
        \item \textit{Qualitative}: \textit{string} que representa a classificação qualitativa do intervalo.
        \item \textit{Half Tones}: número inteiro, representa a quantidade de semitons característica do intervalo.
        \item \textit{Ascendant}: booleano que indica se o intervalo é ou não ascendente, no caso de intervalos melódicos.
      \end{enumerate}

      Excetuando-se métodos de impressão de atributos, a classe possui os seguintes métodos:

      \begin{enumerate}
        \item \textit{Is Dissonant}: retorna um booleano definindo se o intervalo é ou não dissonante, baseado nas regras do Contraponto Palestriniano.
        \item \textit{Is Consonant}: retorna um booleano definindo se o intervalo é ou não consonante.
        \item \textit{Interval to Note}: método estático que retorna uma nota ao receber uma nota e um intervalo. A nota recebida é considerada a primeira ao utilizar a propriedade do intervalo ser ascendente ou descendente, pois uma nota e um intervalo sem atributo de ascendência pode gerar até duas notas, uma acima e outra abaixo da nota original.
      \end{enumerate}

      A Figura \ref{intervalclass} representa os atributos e métodos da classe.

      FIGURA INTERVAL CLASSE

  \section[Módulo de Escalas]{Módulo de Escalas}

    O módulo de escalas é responsável pela geração e armazenamento de escalas musicais. Ele possui uma classe, a \textit{Scale}.

    \subsection[\textit{Scale}]{\textit{Scale}}

    A classe \textit{Scale} armazena as notas de uma escala, possuindo apenas um atributo, um \textit{set} de \textit{strings} chamado \textit{Permitter Notes} que define as notas permitidas para aquela escala. Além disso, ela possui os seguintes métodos:

    \begin{enumerate}
      \item \textit{Is Valid Note}: recebe uma nota e diz se ela está ou não presente na escala.
      \item \textit{Interval To Note On Scale}: método estático que retorna uma nota, dado uma nota, um intervalo e uma escala. Retorna inválido se a nota não estiver presente na escala.
    \end{enumerate}


    A Figura \ref{scaleclass} representa os atributos e métodos da classe.

    FIGURA SCALE CLASSE

  \section[Módulo de Contraponto]{Módulo de Contraponto}

    O módulo de contraponto é responsável pela geração de contrapontos. Utilizando todas as classes anteriormente citadas, atualmente possui duas classes: \textit{Counterpoint} e  \textit{First Order Counterpoint}.

    \subsection[\textit{Counterpoint}]{\textit{Counterpoint}}

    A classe \textit{Counterpoint} funciona como uma biblioteca e não possui atributos de instância. Inicialmente utilizada para a implementação do contraponto de primeira ordem, atualmente possui o método ad-hoc de geração de contrapontos e o método de análise de notas para inserção no contraponto.

    A solução inicialmente concebida utiliza um algoritmo de seleção aleatório. Iniciando na primeira nota do \textit{cantus firmus}, monta-se um vetor de notas possíveis para o contraponto em seu estado atual, escolhe-se aleatoriamente uma das notas do vetor e a insere no vetor do contraponto, seguindo para a próxima nota. O vetor de notas possíveis era montado de acordo com as regras do contraponto de primeira espécie anteriormente citadas.

    O problema dessa solução era o fato de escolher randomicamente um caminho e não explorar os outros, o que podia levar a um caminho que não levava a uma solução. Para contornar isso, o algoritmo era rodado diversas vezes até se encontrar uma solução. Contudo, em músicas maiores o tempo de execução chegava a dois minutos ou mais, e em músicas impossíveis de se gerar contraponto, o algoritmo entrava em \textit{loop} infinito.

    Atualmente, essa classe é utilizada como classe pai da \textit{First Order Counterpoint} e será pai das classes dos contrapontos de outras espécies. A Figura \ref{counterpointclass} apresenta os métodos da classe.

    FIGURA COUNTERPOINT CLASSE

    \subsection[\textit{First Order Counterpoint}]{\textit{First Order Counterpoint}}

  \section[Experimentos]{Experimentos}
  \subsection[\textit{Solução Ad-hoc}]{Solução Ad-hoc}
  \subsection[\textit{Solução com Busca Completa}]{Solução com Busca Completa}
    \subsection[\textit{Solução com DP}]{Solução com DP}
