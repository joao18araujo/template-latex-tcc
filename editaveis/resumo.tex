\begin{resumo}
 A composição algorítmica consiste na criação de peças musicais por meio de algoritmos computacionais, tendo como foco a automação e a imprevisibilidade. O objetivo desse trabalho é desenvolver um \textit{software} capaz de compor algoritmicamente contrapontos palestrinianos para uma dada melodia. Esse desenvolvimento foi feito por meio de um estudo da teoria musical e das regras de contrapontos do século XVI, seguido por um levantamento de requisitos e pela construção de um protótipo com módulos de notação musical, intervalos, escalas e contrapontos capaz de gerar contrapontos de primeira espécie. A parte final do desenvolvimento consistiu no teste dos módulos implementados na fase de prototipação, implementação dos módulos de contrapontos de segunda a quarta espécie e construção de um módulo de construção do MIDI do \textit{cantus firmus} com o contraponto. A solução utilizada trata a geração de contrapontos como um grafo implícito, utilizando DFS em conjunto com programação dinâmica. A aplicação desenvolvida é capaz de ler um arquivo Lilypond completo, armazenar sua melodia, armazenar e utilizar intervalos e escalas, gerando contrapontos de primeira a quarta espécie e enviando ao usuário um arquivo compactado contendo o arquivo Lilypond, o arquivo MIDI e o PDF com a partitura.

 \vspace{\onelineskip}

 \noindent
 \textbf{Palavras-chaves}: composição algorítmica. busca completa. grafos. DFS. dp. programação dinâmica. contrapontos palestrinianos.
\end{resumo}
