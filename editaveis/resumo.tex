\begin{resumo}
 A composição algorítmica consiste na criação de peças musicais por meio de algoritmos computacionais, tendo como foco a automação e a imprevisibilidade, o uso desse tipo de algoritmo possui diversas aplicações. O objetivo desse trabalho é desenvolver um \textit{software} capaz de compor algoritmicamente contrapontos palestrinianos para uma dada melodia. Esse desenvolvimento foi feito por meio de um estudo da teoria musical e das regras de contrapontos do século XVI, seguido por um levantamento de requisitos e pela construção de um protótipo com módulos de notação musical, intervalos, escalas e contrapontos capaz de gerar contrapontos de primeira espécie e servir de guia para o desenvolvimento dos módulos de contrapontos de segunda a quinta espécie. Inicialmente, utilizou-se de um algoritmo de busca completa, que foi trocado por uma solução que trata a geração de contrapontos como um grafo implícito, utilizando DFS combinada a programação dinâmica. O protótipo desenvolvido é capaz de ler nota em formato Lilypond e armazená-las, armazenar e utilizar intervalos e escalas, gerando contrapontos de primeira espécie.

 \vspace{\onelineskip}

 \noindent
 \textbf{Palavras-chaves}: composição algorítmica, busca completa, grafos, dfs, dp, programação dinâmica, contrapontos palestrinianos.
\end{resumo}
