\chapter[Considerações Finais]{Considerações Finais} \label{c4}
  % TODO: expandir considerações finais
  A partir de regras bem definidas e a representação de música de uma forma digital, foi possível gerar contrapontos palestrinianos de primeira a quarta espécie algoritmicamente. Com isso, pode-se afirmar que o trabalho foi concluído com sucesso. Os módulos de notação musical, escala, intervalo, contraponto e de construção de MIDI estão completos, funcionais e testados.

  \section[Trabalhos Futuros]{Trabalhos Futuros}

    Em relação aos trabalhos futuros, as seguintes tarefas podem ser listadas:

    \begin{enumerate}
      \item Resolver \textit{bugs} da implementação incompleta do contraponto de quarta espécie, ele já é gerado, mas fere algumas regras mais complexas.
      \item Refatorar as classes já implementadas para aumentar a solidez do código.
      \item Implementar os módulo de contrapontos livres e de quinta espécie.
    \end{enumerate}
