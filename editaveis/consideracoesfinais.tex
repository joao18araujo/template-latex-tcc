\chapter[Considerações Finais]{Considerações Finais} \label{c4}

  A partir de regras bem definidas e a representação de música de uma forma digital, foi possível gera contrapontos de primeira espécie palestrinianos algoritmicamente. Embora o \textit{parser} ainda possua caráter básico e o resultado ainda seja impresso no terminal, a primeira parte do trabalho foi concluída com sucesso. Os módulos de notação musical, escala e intervalo estão completos e funcionais. O módulo do contraponto de primeira espécie foi concluído e oferece uma boa base para os contrapontos de outras espécies.

  \section[Trabalhos Futuros]{Trabalhos Futuros}

    Em relação aos trabalhos futuros, as seguintes tarefas podem ser listadas:

    \begin{enumerate}
      \item Resolver \textit{bugs} da implementação incompleta do contraponto de segunda espécie, ele já é gerado, mas fere algumas regras mais complexas.
      \item Testar e refatorar as classes já implementadas para aumentar a solidez do código.
      \item Otimizar o \textit{parser} para receber um arquivo Lilypond completo e inserir em uma cópia dele o contraponto gerado.
      \item Implementar os módulos de contrapontos de terceira, quarta e quinta espécie.
      \item Implementar o módulo de construção do MIDI do \textit{cantus firmus} com o contraponto.
    \end{enumerate}
